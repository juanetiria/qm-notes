\documentclass{article}

\usepackage{kafkanotes}

\title{Lecture Notes on Quantum Mechanics}
\author{Daniel David Torres Amaris}

\begin{document}

\begin{titlepage}
\thispagestyle{empty}
\maketitle

\begin{abstract}
Lecture notes on Quantum Mechanics for 2024-2 course taught at the Universidad de Pamplona. This document consists mainly of a compilation of my notes after studying the books of Sakurai and Zettili along with the exercises and examples that I used along the course.
\end{abstract}

\tableofcontents
\end{titlepage}

\newgeometry{top=20mm,bottom=25mm,right=80mm,left=20mm}


\section{Introduction}
Before quantum mechanics, everything was explained in terms of particles and waves, separated worlds whose interaction was described by the Lorentz force or by thermodynamics. Only relativistic and microscopic systems were out of the reach of Newtonian physics. Atomic stability, atomic spectra, the photoelectric effect and black body radiation where the final problems of physics. Quantum mechanics has its origins on the attempt to explain the differences observed between the emission spectra of gases and a black body. The first showed a discrete behavior while latter turned out continuous. 
\begin{marginfigure}%
  \includegraphics[width=\linewidth]{figures/lec19_elements}
  \caption{Discrete spectra for various gases.}
  \label{fig:gasesspect}
\end{marginfigure}
A black body (see fig. \ref{fig:blackbodyimg}) consists of a system emits the same amount of energy that it absorbs; a real life realization of a black body can be achieved by metallic box with reflecting walls and one hole. In such a system, the hole behaves as a black body, absorbing all the wavelengths and emitting all of them after multiple reflection over the inner part of the box. Thus, a black body is a perfect absorber and a perfect emitter.
\begin{marginfigure}%
  \includegraphics[width=\linewidth]{figures/blackbodyimg}
  \caption{Physical realization of a black body.}
  \label{fig:blackbodyimg}
\end{marginfigure}
Besides this notorious differences, further difficulties arose from the very description of the black body radiation. Experimental measurements performed on a variety of black bodies made out of different materials showed energy distributions with zero radiance for the 0 nm wavelength, then increasing until a maximum after which the radiance decreases monotonically until it becomes vanishingly small for large wavelengths. The solid curve in fig. \ref{fig:blackbodyrad} depicts the mentioned behavior.
\begin{figure}%
  \includegraphics[width=\linewidth]{figures/blackbody}
  \caption{Black body radiance distribution.}
  \label{fig:blackbodyrad}
\end{figure}
Attempts to describe the observed behavior where made by Wien and by Rayleigh (see doted and dashed plots respectively in fig. \ref{fig:blackbodyrad}). The Wien's law, derived the spectral radiance $u(\lambda,T)$ from the thermodynamic point of view, achieves a good approximation for small wavelengths and discrepancies for high wavelengths, according to the following equation
\begin{equation}
  u(\lambda, T) = \frac{2hc^2}{\lambda^5} e^{-\frac{hc}{\lambda k_\text{B} T}},
\end{equation}
where $\lambda$ is the wavelength, T is the temperature in Kelvin, $h$, $k_B$ and $c$ are the Planck's, Boltzmann's and the speed of light constants.
The Rayleigh-Jeans distribution, derived from statistics considering the cavity full of standing waves at equilibrium and using the equipartition theorem, arrived to a clear impossibility where the system can have an infinite total energy and infinite radiance is allowed for small wavelengths! 
\begin{equation}
  u(\lambda,T) = \frac{2ck_\text{B}T}{\lambda^4}.
\end{equation}
Planck in an attempt to improve the fitting of the Wien's distribution used a similar procedure than Rayleigh's. Thus, starting from the average energy for the harmonic oscillators inside the cavity
\begin{equation}\label{eq:equipart}
  <E> = \frac{\int_0^\infty E e^{-E/k_B T}dE}{\int_0^\infty e^{-E/k_B T}dE},
\end{equation}
but postulating discrete emitting harmonic oscillators allowed only to emit energy in multiples of $h\nu$,
\begin{equation}
  E = nh\nu, n=0,1,2,...
\end{equation}
he was able to change the integral in eq. \ref{eq:equipart} for a summation as follows
\begin{equation}\label{eq:planckequipart}
  <E> = \frac{\sum_{n=0}^\infty nh\nu e^{-nh\nu/k_B T}}{\sum_{n=0}^\infty e^{-nh\nu/k_B T}} = \frac{h\nu}{e^{\frac{h\nu}{k_B T}}-1}.
\end{equation}
Multiplying eq \ref{eq:planckequipart} by the number of modes allowed inside the cavity, we have:
\begin{equation}\label{eq:planckrad}
  u(\nu,T) = \frac{8\pi\nu^2}{c^3} \frac{h\nu}{e^{\frac{h\nu}{k_B T}}-1}.
\end{equation}
Notice that you can find the Wien's and Rayleigh's distributions in terms of the frequency by using the substitution $\lambda^{-1}=\frac{\nu}{c}$. This calculation is left as an exercise for the student.
Direct integration of eq. \ref{eq:planckrad} over all the whole spectrum gives
\begin{equation}\label{eq:plancktotal}
  E_{Total} = \int_{0}^{\infty}u(\nu,T)d\nu = \frac{8\pi h}{c^3} \int_0 ^{\infty} \frac{\nu^3}{e^{\frac{h\nu}{k_B T}}-1}=\frac{8\pi^5k_B ^4}{15h^3 c^3}T^4=\frac{4}{c}\sigma T^4,
\end{equation}
where $\sigma$ is the Stefan-Boltzmann constant. In eq. \ref{eq:plancktotal} the total energy is not longer infinite but always depending on the temperature as expected. Furthermore, eq. \ref{eq:planckrad} fitted perfectly with the experimental data. Thus, Planck's distribution for black body radiation solved both the ultraviolet catastrophe and worked seamlessly both in the low and high frequency ranges. All thanks to the discretization imposed over the energy emitted by the oscillators. 
In what follows, we are going to see how this idea was taken by other scientists to solve the remaining problems.
\subsection{Exercises}
\begin{itemize}
  \item Calculate the number of modes allowed inside a cubic cavity of volume v=$a^3$ 
  \item Find the expressions for the Wien's and Rayleigh-Jeans' distributions in terms of the frequency
  \item 
\end{itemize}

\section{Compton effect}
The compton effect consist of a shift in wavelength observed for x-ray or $\gamma$-ray photons that scatters away from electrons initially at rest. This interaction provides evidence for the particle behavior of the light; here, the photon is not absorbed, instead, it collides with the electron. After colliding, the electron recoils making the photon lose some of its energy and momentum which is gained to the electron. Therefore, the wavelength of the photon is increased. In fig. \ref{fig:compton} an illustration of process is shown
\begin{marginfigure}%
  \begin{tikzpicture}
    \node[anchor=south west,inner sep=0] at (0,0) {\includegraphics[width=\textwidth]{figures/compton.pdf}};
    \node at (2.3,10.2) {$\gamma: $ $E=h\nu$, $\vec{p}=\frac{h\nu}{c}$};
    \node at (4.5,0.4) {\resizebox{0.45\textwidth}{!}{$\gamma': $ $E'=h\nu'$, $\vec{p'}=\frac{h\nu'}{c}$}};
    \node at (4.5,5.6) {\resizebox{0.43\textwidth}{!}{$e^-:$ $E_e =m_ec^2$, $\vec{p}_e=0$}};
    \node at (5.1,2.7) {\resizebox{0.26\textwidth}{!}{${e^-}':$ $E'_e$, $\vec{p'}_e$}};
    \node at (3.5,4.5) {$\theta$};
  \end{tikzpicture}
  \caption{Compton effect: a photon scatters away from an electron, suffering a change in its wavelength along the process.}
  \label{fig:compton}
\end{marginfigure}
Momentum conservation law can be written as
\begin{equation}\label{eq:momentumconserv}
  \vec{p}+\vec{p}_e=\vec{p'}+\vec{p'}_e,
\end{equation}
but, the electron is initially at rest
\begin{equation}\label{eq:momentumconserv1}
  \vec{p}=\vec{p'}+\vec{p'}_e.  
\end{equation}
Thus, the explicit equation for the electron's momentum after the collision is
\begin{equation}\label{eq:momentumconserv2}
  \vec{p'}_e=\vec{p}-\vec{p'}\rightarrow{p'}^2_e=p^2+{p'}^2-2pp'\cos\theta,
\end{equation}
which, in terms of the frequency results as
\begin{equation}\label{eq:momentumconserv3}
  {p'}^2_e=\left(\frac{h\nu}{c}\right)^2+\left(\frac{h\nu'}{c}\right)^2-2\left(\frac{h\nu}{c}\right)\left(\frac{h\nu'}{c}\right)\cos\theta
\end{equation}
or
\begin{equation}\label{eq:momentumconserv4}
  {p'}^2_e=\frac{h^2}{c^2}\left(\nu^2+\nu'^2-2\nu\nu'\cos\theta\right).
\end{equation}

On the other hand, the energy conservation law demands 
\begin{equation}\label{eq:energyconserv}
  E+E_e=E'+E'_e.
\end{equation}
The energy momentum relation for the electron
\begin{equation}\label{eq:energyconserv1}
  E_e^2 = (p_e \textrm c)^2 + \left(m_e \textrm c^2\right)^2=h\sqrt{\nu^2+\nu'^2-2\nu\nu'\cos\theta +\frac{m_e^2c^4}{h^2}}
\end{equation}
in combination with the momentum of the electron after the collision (eq. \ref{eq:momentumconserv4}) results in 
\begin{equation}\label{eq:energyconserv2}
  h\nu + m_ec^2 = h\nu' + h\sqrt{\nu^2+\nu'^2-2\nu\nu'\cos\theta +\frac{m_e^2c^4}{h^2}}.
\end{equation}
which can be rewritten as
\begin{equation}\label{eq:energyconserv3}
  \nu - \nu' + \frac{m_ec^2}{h} = \sqrt{\nu^2+\nu'^2-2\nu\nu'\cos\theta +\frac{m_e^2c^4}{h^2}}.
\end{equation}
and removing the squared root we obtain
\begin{equation}\label{eq:energyconserv4}
  \nu^2 - \nu'^2 -2\nu\nu' +2(\nu-\nu')\frac{m_ec^2}{h} + \frac{m^2_ec^4}{h^2}= \nu^2+\nu'^2-2\nu\nu'\cos\theta +\frac{m_e^2c^4}{h^2}.
\end{equation}
After some simplification process it remains
\begin{equation}\label{eq:energyconserv5}
-2\nu\nu' +2(\nu-\nu')\frac{m_ec^2}{h} = -2\nu\nu'\cos\theta.
\end{equation}
Rearranging the terms
\begin{equation}\label{eq:energyconserv6}
  \nu\nu'\cos\theta-\nu\nu' + (\nu-\nu')\frac{m_ec^2}{h} = 0,
\end{equation}
or
\begin{equation}\label{eq:energyconserv7}
  \cos\theta-1 + \frac{\nu-\nu'}{\nu\nu'}\frac{m_ec^2}{h} = 0,
\end{equation}
which rewritten again is left as
\begin{equation}\label{eq:energyconserv8}
  \frac{h}{m_ec^2}(\cos\theta-1) + \frac{1}{\nu'}-\frac{1}{\nu} = 0,
\end{equation}
or
\begin{equation}\label{eq:energyconserv9}
  \frac{1}{\nu'}-\frac{1}{\nu} = (1-\cos\theta)\frac{h}{m_ec^2},
\end{equation}
The variable in eq. \ref{eq:energyconserv9} can be changed to in terms of the wavelength
\begin{equation}\label{eq:energyconserv10}
  \Delta\lambda=\lambda'-\lambda = (1-\cos\theta)\frac{h}{m_ec},
\end{equation}
The constant $\frac{h}{m_ec}$ is known as the Compton's wavelength and is denoted by $\lambda_C$=2.426$\times$10$^{-12}$m. Thus, using the $\frac{\theta}{2}$ identity, we have the expression for the Compton shift as follows
\begin{equation}\label{eq:energyconserv11}
  \Delta\lambda=\lambda'-\lambda = 2\lambda_C\sin\frac{\theta}{2}.
\end{equation}

The Compton effect is proof of the wave-particle duality exhibited by the light, a foundational principle of quantum mechanics.

\section{Specific heat of a solid and a diatomic gas}



\end{document}